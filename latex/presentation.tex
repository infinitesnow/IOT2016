\documentclass[11pt]{article}
\title{\textbf{Real data collection}\\
				\large Project for the Internet of Things course @PoliMi Como}
\author{Emanuele Dalla Longa}
\date{2017/11/06}

\usepackage{graphicx}
\usepackage{listings}
\usepackage{amsmath}
\usepackage{hyperref}
\usepackage{textcomp}
\usepackage{multirow}
\usepackage{makecell}
\usepackage{wrapfig}
\graphicspath{{images/}}
\begin{document}

\maketitle

\section{Introduction}
The aim of this project was to program a TelosB hardware to collect temperature and humidity data from the environment, broadcast them over the internet using \texttt{node-red}, analyze them and send the stats via mail. All the code used is available in \href{https://github.com/infinitesnow/IOT2016}{this repository} on GitHub.

\section{Chosen architecture}

\begin{wrapfigure}{r}{0.3\textwidth}
\includegraphics[width=0.28\textwidth]{setup}
\caption{My setup}
\label{fig:setup}
\end{wrapfigure}

My setup is shown in Figure \ref{fig:setup}. I decided to use my Raspberry Pi as a gateway. TelosB was connected via USB, and configured to broadcast its output over the serial port. \texttt{node-red} was configured to publish to two online services: Thingspeak (shown in Figure \ref{fig:thingspeak}) and PubNub. From the PubNub channel, I created a Freeboard dashboard to visualize the broadcasted data, as seen in Figure \ref{fig:freeboard}. \texttt{node-red} was configured to send me and a family member a daily report with minimum, maximum and average for each acquired signal in the previous day. All data have been logged into CSV files, which allowed me to plot the data using \texttt{matplotlib}.

\begin{figure}
\includegraphics[width=\textwidth]{thingspeak}
\caption{Thingspeak channel}
\label{fig:thingspeak}
\end{figure}

\begin{figure}
\includegraphics[width=\textwidth]{freeboard}
\caption{Freeboard.io dashboard}
\label{fig:freeboard}
\end{figure}

\section{Reading the serial port}
To read from the serial port, I tried to manually create a script using Python's \texttt{Serial} module, which read chunks of 21 bytes from the input. This didn't work well, since it caused unexpected readings. Splitting on the \texttt{0x7E} flag byte, which is what the TinyOS library provided in \texttt{tinyos.py} file does, seemed to solve the problem. I switched to \texttt{node-red-node-serialport}, which makes this easy to do and integrates cleanly with \texttt{node-red}. But I still had problems, because when a reading had, usually in its least significant byte, the value of the flag byte, the packet got prematurely truncated and the readings went off scale. So, I modified the Python script adding checks on the packet header and footer to ensure correct aligning, and fixed the reading length.  
 
Our payload is composed by three \texttt{uint16} sensor readings. Here is the relevant part of the parsing done with Python's \texttt{Serial} module:

\begin{lstlisting}
data_array=struct.unpack_from('>BBBHHBBBHHHHB',data_raw)
\end{lstlisting}

This is described in Table \ref{tab:packet}. \texttt{>} tells Python that TelosB has big-endian byte order. The first byte (\texttt{B}) is always a \texttt{\texttildelow} (\texttt{0x7E}), and so is the last one. We have another \texttt{byte} for the serial packet header (\texttt{0x45}). Then we have the AM Package header, with a \texttt{0x00} \texttt{byte}, which tells us that this is in fact an AM Package, two \texttt{unsigned int}s (\texttt{H}) for the destination and source address, and three \texttt{byte}s for the message length, group ID, and Active Message handler type. The next three \texttt{unsigned int} are our payload, followed by an \texttt{unsigned int} for the footer and a terminator \texttt{byte}.

\begin{table}[h]
\centering

\begin{tabular}{ | c | c | c | }
\hline
\texttt{0x7E}	& \texttt{\texttildelow}				& \multirow{2}{*}{\makecell{Serial packet\\header}}	\\ \cline{1-2}
\texttt{0x45}	& \texttt{E}							& 													\\ \hline	
\texttt{0x00}	& Defines that this is an AM package	& \multirow{6}{*}{\makecell{AMPackage\\header}}		\\ \cline{1-2}
\texttt{0xFFFF}	& Source address 						&													\\ \cline{1-2}
\texttt{0x0000}	& Destination address					& 													\\ \cline{1-2}
\texttt{0x??}	& Message length						& 													\\ \cline{1-2}
\texttt{0x??}	& Group ID								& 													\\ \cline{1-2}
\texttt{0x??}	& Handler ID							& 													\\ \hline
\texttt{0x????}	& Temperature							& \multirow{3}{*}{Payload}							\\ \cline{1-2}
\texttt{0x????}	& Humidity								& 													\\ \cline{1-2}
\texttt{0x????}	& Light intensity						& 													\\ \hline
\texttt{0x????}	& Footer								& Footer											\\ \hline
\texttt{0x7E}	& \texttt{\texttildelow}				& \makecell{Serial package\\terminator}				\\ \hline
\end{tabular}

\caption{Serial packet structure}
\label{tab:packet}
\end{table}

\section{Sensor calibration}
The sensors I needed to use were a Sensirion SHT11, which provides humidity and temperature readings, and a Humamatsu S1870, which provides visible light intensity. 

I got the transfer function coefficients for the Sensirion sensor from the vendor's datasheet. For the humidity, we have:
$$\text{humidity}=c_1+c_2\cdot x +c_3\cdot x^2$$
where
$$c_1= -2.0468$$
$$c_2=  0.0367$$
$$c_3= -1.5955\cdot10^{-6}$$
for the temperature, we have
$$\text{temperature}=d_1+d_2\cdot x$$ 
$$d_1= -40$$
$$d_2=  0.01$$

Then, I got the Humamatsu conversion factors from the TinyOS wiki. The light sensor uses the microcontroller's 12-bit ADC. We convert the raw value of the ADC to the corresponding voltage like this:

$$ V_{sensor}=\frac{V_{raw}}{4096} * V_{ref} $$ 

Then, we need the current created by the photodiode through a 100kOhm resistor:

$$I = \frac{ V_{sensor} }{10^5}$$

Finally, the value in lux is calculated using the vendor's datasheet: 

$$\text{light intensity} = 0.625\cdot10^6 \cdot I \cdot 1000$$
  

\section{Sensor data}

We are now going to plot some data acquired from the TelosB board.

In Figure \ref{fig:lamp} we see a test plot made on the evening of June 13$^{th}$. We can see the effect of turning a light bulb lamp on, shedding light directly on the sensor. We see light intensity peaking, temperature rising and humidity dropping.


\begin{figure}[h]
\includegraphics[width=\textwidth]{lamp}
\caption{Lamp activation pattern}
\label{fig:lamp}
\end{figure}

Figure \ref{fig:june14} is the plot of June 14$^{th}$ activity for the whole day. We see clearly the lamp activation pattern twice. Also, it started raining around 6PM, and we can see humidity rising up heavily while temperature drops a little.

\begin{figure}[h]
\includegraphics[width=\textwidth]{log-20170514}
\caption{June 14$^{th}$ data}
\label{fig:june14}
\end{figure}

Figure \ref{fig:june14light} shows the plot for the light intensity on June 14$^{th}$, with the $y$ axis limit set to show the data shadowed by the lamp activation. We see that the light stopped rising around noon, when the sky got cloudy. 

\begin{figure}[h]
\includegraphics[width=\textwidth]{log-20170514-light}
\caption{June 14$^{th}$ light intensity plot, with limit on $y$ axis set to 150 lux}
\label{fig:june14light}
\end{figure}

Figure \ref{fig:june15} and Figure \ref{fig:june16} show data for June 15$^{th}$ and June 16$^{th}$.


\begin{figure}[h]
\includegraphics[width=\textwidth]{log-20170515}
\caption{June 15$^{th}$ data}
\label{fig:june15}
\end{figure}

\begin{figure}[h]
\includegraphics[width=\textwidth]{log-20170516}
\caption{June 16$^{th}$ data}
\label{fig:june16}
\end{figure}

\section{Forwarding report via mail}
What I did then was to create a report from the data, which are logged in a \texttt{.csv} file; this is read once a day, parsed, and from it are computed the max, min, and average of the sampled data sources. For the light, considering the lamp pattern, I decided to add a trimmed average, which excludes from this average samples which are more than $3\sigma$ far from the common average. The report is then sent by mail at 13:00 PM.

\begin{figure}[h]
\includegraphics[width=\textwidth]{summary}
\caption{Screenshot of the mail received on June 17$^{th}$, with June 16$^{th}$ data}
\label{fig:summary}
\end{figure}

Due to the limited hardware resources of the Raspberry Pi, computing these stats is quite computationally intensive. Being the machine single core/single thread, and being the Python sampler spawned in another process, when it's computing the stats the flow is halted, some packages are stuck in the reading queue and are read in one single shot, causing the JSON parser to thow an exception. This exception is caught and handled with a splitter. Notice that another JSON parser is used, so that further exceptions are not caught. This looks safer, although it introduces a run condition for which a sample may be posted in between the ones stuck in the queue, thus altering the order. This may swap at most 2 packets/day, so its impact is negligible.

In Figure \ref{fig:summary} is shown an example of a received mail report.

\end{document}
